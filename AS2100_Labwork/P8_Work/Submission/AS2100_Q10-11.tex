\documentclass[12pt,a4paper]{article}	%only 10 to 12 working in article
\usepackage[left=2cm, right=2cm, top=1cm]{geometry}

\usepackage{upgreek}
%type greek letter command between $' '$ !!


\title{10. Best Estimate of Shear Modulus from all Non-Anomalous Experiments}
\date{\vspace{-5ex}}	%to skip printing the dat in the title

\begin{document}
\maketitle
\thispagestyle{empty}	%To supress the printing of page number
\large
For the Best estimate of the Shear Modulus G we find mean($\mu$) and standard deviation($\sigma$) from experimental values of G obtained from all Experiments (i.e Exp1, Exp2, Exp3, Exp4) except Exp5 since it is Anomalous\\

\textbf{Experimental Result : }

Mean($\mu$) = 20.2659 GPa

Standard Deviation($\sigma$) = 3.2629 GPa\\

\textbf{Experimental Value of Shear Modulus : 20.2659 $\pm$ 3.2629 GPa} \\

\textbf{Published Value of G = 26 GPa}

\vspace{2cm}
\begin{center}
	{\LARGE 11. Short Comings and Alternatives}
\end{center}
\Large
While looking at our dataset for theta, we noticed that many a times during loading or unloading the value of theta was negative. During the addition of incremental weights, the apparatus may have been allowed to unload completely as the experimenter could have been holding up the weights, causing the anomalies we see in the data.\\

Since in the experiment,we use the Masses and the Torsion Appratus instead of machines to do the same, The Errors in all the manual machines used adds up to the error in the value of Shear Modulus obtained due to Error Propogation.\\
\thispagestyle{empty}	%To supress the printing of page number
\pagebreak

Furthermore we are using lever and mass to produce torque but also causes a downward force to act on the rod which adds a bending moment to it. According to our assumption we assume the rod to be under pure torsion. Thus producing the Torque using a Chuck would eliminate the external bending moment to a large extent (although Bending Moment due to the weight of the rod still persists)\\


From the analysis of our data, we propose the following to outperform the current experiment.\\
\begin{itemize}
\item A better experiment would be using an improved apparatus such as a torsion testing machine that is equipped with a chuck to provide the torque and offer greater precision. \\

\item Furthermore, regularly changing the testing sample after each cycle or a certain number of cycles to ensure that data remains unaffected due material fatigue can also give a better data reading.\\

\item It is seen that during the experiments, the aluminium rod would regularly go beyond the limit of proportionality of the material\\

To keep a check on when the Data goes beyond the Proportionality Limit, we can reduce the value of $\Delta$m and get more data points so that we can plot a better curve and find out where the material and data points exceed the Proportionality Limit of the given Material\\
\end{itemize}
\thispagestyle{empty}	%To supress the printing of page number
\pagebreak
\thispagestyle{empty}	%To supress the printing of page number
Other method that can be used to find the Shear Modulus of a material is:\\

\begin{itemize}
\item \textbf{Rail Shear Test} \\
This is a very popular method used to measure in-plane shear properties. This method is extensively used in aerospace industry. The shear loads are imposed on the edges of the laminate using specialized fixtures.\\
There are two types of such fixtures:

\begin{enumerate}	
	\item  \textbf{Two Rail Fixture}\\
	The two rail shear test fixture has two rigid parallel steel rails for loading purpose. The rails are aligned to the loading	direction and load (compressive or tensile) is applied to it.\\
	
	Another modification made to it is the \textbf{ V-Notched Rail Shear Test}.\\
	
	2 Strain Gauges are attached to the centre of the V, such that they measure strain along the $\pm$ 45$^{\circ}$ axis. Furthermore,the positioning tools offer high reproducibility.\\
	
	
	\item \textbf{Three Rail Fixture}(improved version)\\
	Using one more rail in two rail	shear test fixture it can produce a closer approximation to pure shear. The fixture consists of 3 pairs	of rails clamped to the test specimen\\
	
	The outside pairs are attached to a base plate which rests on the test machine. Another pair (third middle) pair of rails is guided through	a slot in the top of the base fixture. The middle pair loaded in compression\\
	
	
\end{enumerate}

\item \textbf{Four-Point Loading or Saddle Test}

Four-point loading or saddle test is used to determine the shear modulus of a material. To accomplish this, we apply equal load force on the four corners of a square plate perpendicular to the plate. Forces along one of the diagonals face upwards while the forces along the other diagonal will be facing downward, which will result in a deflection of the plate that looks similar to a saddle.\\

The vertical cross-sections along the diagonal will result in two identical parabolas, one concave upwards and the other concave downwards.
The shear modulus can is calculated from the ratio of the imposed loads on the plate and the vertical deflection of the plate with respect to its geometric centre.\\

\[ G = \frac{3Pu^2}{2wt^3}\]



P = Load applied at each corner\\
t = plate thickness\\
w = deflection of a point (x,y) on the diagonal with respect to the center of the plate\\
u = the Diagonal Distance from the Center to the point (x,y)\\
G = Shear Modulus \\

\end{itemize}
The Advantage in the above Tests is that the increment in stress is almost continuous and gives a very precise,reproducible plot with minimal error.\\

There Error Propogation is also diminshed owing to the use of advanced machinery which gives highly precise and accurate readings.\\

\thispagestyle{empty}	%To supress the printing of page number
\end{document}
