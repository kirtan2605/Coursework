\documentclass[12pt,a4paper]{article}	%only 10 to 12 working in article
\usepackage[left=2cm, right=2cm, top=0.5cm]{geometry}


\usepackage{upgreek}
%type greek letter command between $' '$ !!

\title{2. Sources and minimization of errors in the experiment}
\date{\vspace{-5ex}}	%to skip printing the dat in the title
\begin{document}
\maketitle
	\section{Systematic Error}This type of error arises due to defect in the measuring device.It is repetitive in nature and can be removed after being calculated to get a precise value of the measurement.\\
		\textbf{1.} The dimension of the specimen that has been measured may not accurate. Zero error may be occurred on the Vernier caliper or the Torsion Test Apparatus. It can be subtracted from the measured reading to obtain true value.\\
		\textbf{2.} The blocks of mass used for the experiment might have some error.They should be checked and corrected before the experiment to minimize error due to them
	\section{Random Error}This type of error could occur due to sudden change in experimental conditions. The specimen could have	exposed to undesired and unsuitable temperature and humidity. It is an accidental error and is beyond our control. Random errors, unlike systematic errors, can often be quantified by statistical analysis, therefore, the effects of random errors on the quantity or physical law under investigation can often be determined.\\
	\\They are mostly positive and negative fluctuations that cause about one-half of the measurements to be too high and one-half to be too low. Sources of random errors cannot always be identified. Possible sources of random errors are as follows:
	
	\textbf{1.Observational} For example, errors in judgment of an observer when reading the scale of a measuring device to the smallest division.	
	
	\textbf{2. Environmental} For example, unpredictable fluctuations in line voltage, temperature, or mechanical vibrations of equipment.
	
	Repeated measurements produce a series of data sets that are all slightly different. They vary in random vary about an average value. Taking many sets of reading can minimize the random errors
	
	\section{Human Error} This type of error could occur due to the faulty procedure	adopted by us. A person may record a wrong value, misread a scale(including Parallax Error), forget a digit when reading a scale or recording a measurement, or make a similar blunder.
	\\These errors can be reduced if there are multiple different people conducting the experiment.It will reduce the Human error if not completely remove it.\\
 	\line(1,0){500}\\
	\textbf{1.}When apply the load we must be careful otherwise on additional force will occur and it cause to the error readings.\\
	\textbf{2.}This apparatus has been used for a long time for this experiment so the rod has been subjected to torque many times. It may be subject to fatigue.To avoid this, the rod must be changed at regular interval.
		
	\thispagestyle{empty}	%To supress the printing of page number
	
	\pagebreak
	
	Many parameters may be expected to influence the accuracy of this test method. Some of these parameters pertain to the uniformity of the specimen, for example, its straightness, the uniformity of its diameter, and, in the case of tubes, the uniformity of its wall thickness.\\
	
	The variation in shear modulus $\Delta$G due to variations in diameter $\Delta$ D are given by:
	
	\[ \frac{\Delta G}{G} = -4 \frac{\Delta D}{D}\] \\

	
	Other parameters that may be expected to influence the accuracy of this test method pertain to the testing conditions, for example, alignment of the specimen, speed of testing, temperature, and errors in torque and twist values.\\
	
	The error in shear modulus $\Delta$G due to errors in torque $\Delta$T are given by:
	
	\[ \frac{\Delta G}{G} = \frac{\Delta T}{T} \] \\
	
	
	
	12.3.2 According to Eq 2 (see 6.1), the error in shear modulus $\Delta$G due to errors in angle of twist $\Delta$ $\theta$ are given by:
	
	\[ \frac{\Delta G}{G} = \frac{\Delta \theta }{\theta} \] \\
	
	
	
	The least count of the twist gage should always be smaller than the minimum acceptable value of $\Delta$ $\theta$. In general, the overall precision that is required in twist data for the determination of shear modulus is of a higher order than that required of strain data for determinations of most mechanical properties, such as yield strength. It is of the same order of precision as that required of strain data for the determination of Young's modulus .

\thispagestyle{empty}	%To supress the printing of page number
\end{document}