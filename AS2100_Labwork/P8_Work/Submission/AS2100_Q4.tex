\documentclass[12pt,a4paper]{article}	%only 10 to 12 working in article
\usepackage[left=2cm, right=2cm, top=1cm]{geometry}

\usepackage{upgreek}
%type greek letter command between $' '$ !!

\title{4. Effect of changing the number of cycles per experiment in the mean value of G and Effect of mass increment value $\Delta$m in accuracy or precision of the G for a given
m\textsubscript{max}}
\date{\vspace{-5ex}}	%to skip printing the dat in the title

\begin{document}
\maketitle
Accuracy is how close a measurement is to the correct value for that measurement.The precision of a measurement system is refers to how close the agreement is between repeated measurements (which are repeated under the same conditions).\\ \\
Precision is sometimes separated into :\\
    • Repeatability — The variation arising when all efforts are made to keep conditions constant by using the same instrument and operator, and repeating the measurements during a short time period. \\
    • Reproducibility — The variation arising using the same measurement process among different instruments and operators, and over longer time periods. 
\\
\\
The random error will be smaller with a more accurate instrument (measurements are made in finer increments) and with more repeatability or reproducibility (precision). \\
\\
In our experiment we determine the Shear Modulus of Aluminium 6061 by performing the Torsion Test. The experiment is carried and a value for G is obtained in that cycle. Just to be on the safe side, the experiment is being repeated (i.e multiple cycles are performed). It is highly unlikely that the second cycle will yield the same result as the first. In fact,if we run a number of replicate (that is, identical in every way) trials, we will probably obtain scattered results.\\
\\
As stated above, the more measurements that are taken, the closer we can get to knowing a quantity’s true value. With multiple measurements (replicates), we can judge the precision of the results, and then apply simple statistics to estimate how close the mean value would be to the true value if there was no systematic error in the system. The mean deviates from the “true value” less as the number of measurements (replicates) increases.
Thus,if we increase the number of cycles per experiment ,the mean value of G obtained will be closer to the true value of G.\\
\\
Yes,mass increment value $\Delta$m effects the accuracy and precision of the G for a given m\textsubscript{max}.\\
\\
\textbf{1.}If we have a smaller value of $\Delta$m ,We can plot the Stress vs Strain curve better and hence do not take those points into the calculation of G which are beyond the proportionality limit (IF The m\textsubscript{max} corresponds to a strain which is beyond the proportionality limit). In this case it affects the accuracy of G.\\
\\
\textbf{2.}Taking smaller value of delta m will reduce the random error caused in while taking measurements and hence will make our experimental value  of G more precise. Thus, $\Delta$m will always affect the precision of G.
\thispagestyle{empty}	%To supress the printing of page number
\end{document}
